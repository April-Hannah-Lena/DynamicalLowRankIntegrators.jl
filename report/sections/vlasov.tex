% -------------------------------------------------------------------------------------- %

\section{The Vlasov-Poisson Equation and Conservation Properties}\label{sec:vlasov}

The Vlasov-Poisson equation models the evolution of the electron density $f = f(x, v, t)$ 
in plasma. It is assumed that $(1)$ slow movement of heavy positive ions, $(2)$ pair 
collisions between electrons, $(3)$ variations in magnetic field, and $(4)$ relativistic 
effects are all negligible. The resultant (non-dimensionalized) evolution equation reads

\begin{equation}\label{eq:vlasov}
    \partial_t f (x, v, t) 
    + v \cdot \nabla_x f (x, v, t) 
    + E (x, t) \cdot \nabla_v f (x, v, t) = 0
\end{equation}

where $E = E(f)$ is the self-consistent electric field. The mass 

\begin{equation}
    \scrM_0 (t) = \int_{\Omega_x} \int_{\Omega_v} f(x, v, t)\ dv\, dx
\end{equation}

is wlog. assumed to be initially equal to $1$. Moreover $E$ is written as a potential 
$E = - \nabla \phi$ where 

\begin{equation}
    - \Delta \phi\,(x) = 1 - \int_{\Omega_v} f(x, v, t)\ dv . 
\end{equation}

Equation \ref{eq:vlasov} contains infinitely many invariants. Indeed, all $L^p$ norms 
$\left\| f \right\|_p$ as well as entropy 

\begin{equation}
    \mathcal{S} (t) = \iint_{\Omega_x \times \Omega_v} f(x, v, t)\ \log f(x, v, t)\ dv\,dx
\end{equation}

are conserved \cite{}\todo{cite entropy conservation}. Additionally it can be shown 
that all moments of the velocity distrbution

\begin{equation}
    M_k (x, t) = \int_{\Omega_v} v v \ldots v f(x, v, t)\ dv 
\end{equation}

(wich are $k$-tensors) satisfy continuity equations and correspondingly their 
space-integrals $\scrM_k = \int_{\Omega_x} M_k\, dx$ are also conserved. Specifically, 
for a cartesian index 
$\iota = (\iota_1,\, \ldots,\, \iota_k) \in \left\{ 1, \ldots, d \right\}^k$ of $M_k$ 
there is a multiindex $\varrho = \left( \varrho_1, \ldots \varrho_d \right) \in \bbN^d$ with 
$| \iota | = k$ and 

\begin{equation}\label{eq:multiindex}
    M_k^\iota = \int_{\Omega_v} v_{\iota_1} v_{\iota_2} \ldots v_{\iota_k}\, f \,dv
    = \int_{\Omega_v} v_1^{\varrho_1} v_2^{\varrho_2} \ldots v_d^{\varrho_d}\, f \,dv . 
\end{equation}

The continuity equation for $M_k^\iota$ is written 

\begin{equation}
    \partial_t \int_{\Omega_v} v^\varrho f \,dv 
    + \nabla_x \cdot \int_{\Omega_v} v^{\varrho + 1} f \,dv
    + E \cdot \int_{\Omega_v} \varrho v^{\varrho - 1} f \,dv = 0
\end{equation}

where $z^{\varrho \pm 1}$ is understood as the vector

\begin{equation}\label{eq:continuity}
    z^{\varrho \pm 1} = \begin{bmatrix}
        z^{\left( \varrho_1 \pm 1,\, \varrho_2,\, \ldots,\, \varrho_d \right)} \\
        z^{\left( \varrho_1,\, \varrho_2 \pm 1,\, \ldots,\, \varrho_d \right)} \\
        \vdots \\
        z^{\left( \varrho_1,\, \ldots,\, \varrho_{d-1},\, \varrho_d \pm 1 \right)}
    \end{bmatrix} . 
\end{equation}

Notable are the first few moments and their continuity equations

\begin{align}
    &\partial_t M_0 + \nabla_x \cdot M_1 = 0, \label{eq:mass_continuity}\\
    &\partial_t M_1 + \nabla_x \cdot M_2 + E M_0 = 0, \label{eq:momentum_continuity}%\\
    %&\partial_t M_2 + \nabla_x \cdot M_3 + 2 E \otimes M_1 = 0 \label{eq:second_moment_continuity}
\end{align}

which imply conservation of mass $\scrM_0$ and momentum $\scrM_1$ 
%and (kinetic) energy $\trace \scrM_2$ 
after integrating over $\Omega_x$. The continuity equation for (total) 
energy density is deduced from the trace of the second moment

\begin{equation}\label{eq:energy_continuity}
    \partial_t e + \frac{1}{2} \nabla_x \cdot (\trace M_3) = (\partial_t E - M_1) \cdot E
\end{equation}

where 

\begin{equation}
    e(x, t) = \frac{1}{2} \trace M_2 (x, t) + \frac{1}{2} E(x, t)^2
    = \frac{1}{2} \int_{\Omega_v} v^2 f(x, v, t)\ dv + \frac{1}{2} E(x, t)^2  
\end{equation}

and 

\begin{equation}
    \trace M_3 (x, t) = \int_{\Omega_v} v^2\ v f(x, v, t)\ dv . 
\end{equation}
\todo{this is trace of a 3-tensor ie. tensor contraction - should maybe simplify?} 

The dynamical low-rank integrator presented in the following section is designed to 
satisfy equations \ref{eq:mass_continuity}, \ref{eq:momentum_continuity}, and 
\ref{eq:energy_continuity}. 


% -------------------------------------------------------------------------------------- %
