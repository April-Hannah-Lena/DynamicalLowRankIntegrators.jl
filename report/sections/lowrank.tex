% -------------------------------------------------------------------------------------- %

\section{A Low-rank Tensor Approximation Scheme}\label{sec:lowrank}

\subsection{Presentation of the Integrator}

The density function $f(x, v, t) \in L^2 (\Omega_x \times \Omega_v)$ is approximated by a 
tensor product of functions in $x$ and in $v$:

\begin{equation}
    f(x, v, t) = f_0 (v) \sum_{i, j = 1}^{r} X_i (x, t)\ S_{i j} (t)\ V_j (v, t)
    \eqdef f_0 (v)\ X (x, t)^T \,S (t) \,V (v, t)
\end{equation}

where $r$ is the approximation rank and $f_0 (v) = \exp ( - v^2 )$ is a Gaussian weight. 

Let 

\begin{equation}
    \left\langle g, h \right\rangle_x = \int_{\Omega_x} f\,g\ dx, \quad 
    (g, h)_v = \int_{\Omega_v} g\,h\ dv \quad 
    \left\langle g, h \right\rangle_v = \int_{\Omega_v} g\, h\, f_0\ dv . 
\end{equation}

We require that $X$ and $V$ satisfy the orthonormality conditions 

\begin{equation}\label{eq:orthonormal}
    \left\langle X_i,\, X_j \right\rangle_x = \delta_{i j}, \quad 
    \left\langle V_i,\, V_j \right\rangle_v = \delta_{i j}, \quad 
    1 \leq i, j \leq r 
\end{equation}

where $\delta_{i j}$ refers the the Kronecker delta, as well as the gauge conditions 

\begin{equation}\label{eq:gauge}
    \left\langle \partial_t X_i,\, X_j \right\rangle_x = 0, \quad 
    \left\langle \partial_t V_i,\, V_j \right\rangle_v = 0, \quad 
    1 \leq i, j \leq r . 
\end{equation}

Let further $\bar{X} = \spn \left\{ X_1, \ldots, X_r \right\}$, 
$\bar{V} = \spn \left\{ V_1, \ldots, V_r \right\}$. The Galerkin condition yields the 
equation \cite{einkemmer2018}:

\begin{equation}\label{eq:galerkin}
    \partial_t f = \Pi \left(\, RHS (f) \,\right) 
    \quad \text{where} \quad 
    \Pi\, g = \Pi_{\bar{V}}\, g - \Pi_{\bar{X}}\Pi_{\bar{V}}\, g + \Pi_{\bar{X}}\, g
\end{equation}

and $\Pi_{\bar{X}}$, $\Pi_{\bar{V}}$ are the orthogonal projections onto $\bar{X}$ and 
$\bar{V}$, respectively. A first-order Lie-Trotter splitting based on the three terms in 
equation \ref{eq:galerkin} yields equations of motion for the components of $X$, $S$, 
and $V$. 

The key insight of \cite{einkemmer2021}
is that if the functions $v \mapsto 1$, $v \mapsto v_1, \ldots, v \mapsto v_d$, and 
$v \mapsto v^2$ lie in $\bar{V}$, then discrete versions of equations 
\ref{eq:mass_continuity}, \ref{eq:momentum_continuity}, and \ref{eq:energy_continuity} 
hold. Hence the integration scheme is altered to guarantee this condition. We split 
$V (v, t) \in \bbR^r$ into two blocks $U(v, t) \in \bbR^m$ and $W \in \bbR^{r-m}$: 

\begin{equation}
    V = \begin{bmatrix}
        U \\
        W
    \end{bmatrix}
\end{equation}

where $U$ is fixed throughout the integration and contains the desired functions. To 
satisfy the orthonormality condition \ref{eq:orthonormal} we use Hermite polynomials

\begin{equation}
    U(v, t) = U(v) \propto \begin{bmatrix}
        1 \\
        2v_1 \\
        \vdots \\
        2v_d \\
        4v^2 - 2
    \end{bmatrix} . 
\end{equation}

Performing the analogous calculations as in \cite{einkemmer2018}
for the altered basis functions yields \cite{einkemmer2021}

\begin{align}
    \sum_i \partial_t X_i\, S_{i k} &= 
        (\,V_k,\, RHS(f)\,)_v - \sum_i X_i\, \partial_t S_{i k} ,
        \quad\quad 1 \leq k \leq r \label{eq:X_galerkin}\\
    \sum_{i p} S_{i\, q+m}\, S_{i p}\, \partial_t W_p &= 
        \frac{1}{f_0} \sum_i S_{i\, q+m} \left\langle\, X_i,\, RHS(f) \,\right\rangle_x 
        - \sum_{i l} S_{i\, q+m}\, \partial_t S_{i l}\, V_l ,
        \quad\quad 1 \leq q \leq r - m \label{eq:V_galerkin}\\
    \partial_t S_{k l} &= \left( \,X_k,\, ( V_l,\, RHS(f) )_v\, \right)_x 
        \quad\quad 1 \leq k, l \leq r . \label{eq:S_galerkin}
\end{align}

Concrete equations for computing the above inner products are given in 
\cite{robustlowrank}. It should be reemphasized that the orthonormality and guage 
conditions \ref{eq:orthonormal}, \ref{eq:gauge} must still hold, so the components of $U$ 
must be appropriately scaled. 

To solve equations \ref{eq:X_galerkin}, \ref{eq:V_galerkin} directly, the matrix $S$ must 
be inverted. However, as the approximation rank $r$ increases, $S$ has progressively 
smaller singular values. Hence the scheme becomes increasingly ill-conditioned as the 
accuracy increases. We therefore need to alter the low-rank scheme again to address this 
issue. 

Notice that the low-rank approximation $f = X^T S V$ can be written as $f = K^T V$ for 
some $K = K(x, t) \in \bbR^r$. $X^T$ and $S$ are then (up to a unitary basis 
transformation) the result of a (semidiscrete\footnote{
    The standard Gram-Schmidt process to construct a QR decomposition can just as easily 
    be viewed in a semidiscrete setting: each "column" of $K^T$ is a function of $x$ 
    evaluated across $\Omega_x$, instead of each column being a discrete vector. The 
    algorithm does not need to be changed at all. 
}) QR factorization of $K^T$. Analogously, $f$ can be written as $f = X^T L$ and the 
elements $W$ can also be reconstructed by a QR factorization. We may therefore rewrite 
equations \ref{eq:X_galerkin} and \ref{eq:V_galerkin} as 

\begin{align}
    \partial_t K_k &= (\,V_k,\, RHS(f)\,)_v, \quad\quad 1 \leq k \leq r \label{eq:K_galerkin}\\
    \partial_t L_q &= 
        \frac{1}{f_0} \sum_i S_{i\, q+m} \left\langle\, X_i,\, RHS(f) \,\right\rangle_x 
        - \sum_{i l} S_{i\, q+m}\, \partial_t S_{i l}\, V_l ,
        \quad\quad 1 \leq q \leq r - m \label{eq:L_galerkin} 
\end{align}\todo{mMn sollte die Ableitung im letzten Term eigentlich aud das erste S sein, 
aber so haben die es im Einkemmer Paper gemacht und es functioniert}

with 

\begin{equation}
    K_k = \sum_i X_i\, S_{i k}, \quad\quad L_q = \sum_{i l}\, S_{i\, q+m}\, S_{i l} W_l . 
\end{equation}

Using a stepping scheme

\begin{equation}
    K (t + \tau) = K (t) + \tau \partial_t K (t), \quad\quad
    L (t + \tau) = L (t) + \tau \partial_t L (t)
\end{equation}

we may obtain $X(t + \tau)$, $V(t + \tau)$ 
via QR factorization

\begin{equation}
    K_k (t + \tau) = \sum_i X_i (t + \tau)\, R^1_{i k}, \quad\quad
    L_q (t + \tau) = \sum_i W_i (t + \tau)\, R^2_{i q} . 
\end{equation}

Finally, we compute $S(t + \tau)$ using equation \ref{eq:S_galerkin} and a 
best-approximation of $f$:

\begin{equation}
    f \approx \widetilde{ f } \defeq f_0\, X(t + \tau)^T\, (M^T S N)\, V(t + \tau), 
\end{equation}

\begin{equation}\label{eq:S_step}
    S_{k l} (t + \tau) = \sum_{i j} M^T_{k i}\, S_{i j}\, N_{j l} 
        + \tau \left( X_k,\, \left( V_l,\, 
            RHS( \widetilde{ f } )\, 
        \right)_v \right)_x
\end{equation}

where 

\begin{equation}\label{eq:projection}
    M_{k i} = \left\langle X_k (t),\, X_i (t + \tau) \right\rangle_x, \quad\quad
    N_{j l} = \left\langle V_j (t),\, V_l (t + \tau) \right\rangle_v . 
\end{equation}
\todo{Rename $M$ and $N$ so that it doesn't conflict with moments}

Crucially, the approximation $f \approx \widetilde{ f }$ doe \emph{not} conserve any of 
the invariants since the projections in \ref{eq:projection} are not conservative. 
Therefore we need to expand the basis onto which we project. 

Specifically, let 
$(\widetilde{ X }_j)_j$ be an orthonormal basis of 
$\spn \left\{ X_i (t), \nabla X_i (t), E(t)X_i (t), K_i (t + \tau) \right\}_i$ and 
$(\widetilde{ V }_j)_j$ an orthonormal basis of 
$\spn \left\{ V_i (t), L_q (t + \tau) \right\}_{i q}$. 
\emph{We emphasize that this definition of $\widetilde{ X }$ differs slightly from 
previous works due to the addition of $E X$}. This drastically reduces the complexity of 
future proofs, while only increasing the basis size linearly. 

The projections 

\begin{equation}\label{eq:conservative_projection}
    \widetilde{ M }_{k i} = \left\langle \widetilde{ X }_k,\, X_i (t) \right\rangle_x, \quad\quad
    \widetilde{ N }_{j l} = \left\langle \widetilde{ V }_j,\, V_l (t) \right\rangle_v 
\end{equation}

are mass, momentum, and energy conservative \cite{robustlowrank}. However, this has 
increased the rank of the approximation $f$. Thus, we need to truncate the approximation 
in a way which ensures that the fixed basis functions of $U$ remain unchanged. For 
convenience write 

\begin{equation}
    \widetilde{ S } (t) = \widetilde{ M }^T S(t) \widetilde{ N }
\end{equation}

where $M$ and $N$ are as in equation \ref{eq:conservative_projection}, and write 
$\widetilde{ S } (t + \tau)$ as the result of applying equation \ref{eq:S_step} with 
$\widetilde{ M }$, $\widetilde{ S }(t)$, $\widetilde{ N }$, $\widetilde{ X }$, and $\widetilde{ V }$. 
Letting 
$\widetilde{ K }^T = \widetilde{ X }^T \widetilde{ S } (t + \tau)$ and using the structure of 
$\widetilde{ V }$, 

\begin{equation}\label{eq:K_L_step}
    f (t + \tau) \approx \widetilde{ K }^T \widetilde{ V }
    = \begin{bmatrix}
        (\widetilde{ K }^{cons})^T & (\widetilde{ K }^{rem})^T
    \end{bmatrix}
    \begin{bmatrix}
        U \\ 
        \widetilde{ W }
    \end{bmatrix}
\end{equation}

where $\widetilde{ K }^{cons}$ is the first $m$ components of $\widetilde{ K }$, and 
$\widetilde{ K }^{rem}$, $\widetilde{ W }$ are the last components of 
$\widetilde{ K }$, $\widetilde{ V }$ respectively. 

Hence, by truncating $\widetilde{ K }^{rem}$ and $\widetilde{ W }$, the desired 
components of $U$ remain unaffected. We perform the truncation as follows: QR 
factorizations of $\widetilde{ K }^{cons}$, $\widetilde{ K }^{rem}$ yield 

\begin{equation}
    \widetilde{ K }^{cons}_k = \sum_i X^{cons}_i\, S^{cons}_{ i k} , \quad\quad
    \widetilde{ K }^{rem}_q = \sum_j \widetilde{ X }^{rem}_j\, \widetilde{ S }^{rem}_{j q} . 
\end{equation}

By a truncated singular value decomposition of $\widetilde{ S }^{rem}$, keeping only the 
largest $r - m$ singular values, we have 

\begin{equation}
    \widetilde{ S }^{rem} \approx \widehat{ U } \widehat{ S } \widehat{ W } . 
\end{equation}

Now set $S^{rem} = \widehat{ S }$ and 

\begin{equation}
    X^{rem}_q = \sum_i \widetilde{ X }^{rem}_i\, \widehat{ U }_{i q} , \quad\quad 
    \widecheck{ W }_q = \sum_j \widetilde{ W }_j\, \widehat{ W }_{j q} , 
    \quad\quad 1 \leq q \leq r - m . 
\end{equation}

Combining $\widehat{ X } = \begin{bmatrix} X^{cons} \\ X^{rem} \end{bmatrix}$ and 
performing a final QR factorization

\begin{equation}
    \widehat{ X }_k = \sum_i \widecheck{ X }_i\, R_{i k}
\end{equation}

finishes the truncation, as we set

\begin{equation}
    X (t + \tau) = \widecheck{ X }, \quad\quad 
    S (t + \tau) = R \begin{bmatrix}
        S^{cons} & \\
        & S^{rem}
    \end{bmatrix}, \quad\quad
    V (t + \tau) = \begin{bmatrix}
        U \\[0.5em] 
        \widecheck{ W }
    \end{bmatrix} . 
\end{equation}

While in this case we have performed the time-stepping in equations \ref{eq:K_L_step}, 
\ref{eq:S_step} via a simple explicit Euler scheme, the extension to time steps of higher 
order is immediate. Indeed, we refer to \cite{ceruti2024} for an extension of the robust 
integrator using the midpoint rule. Pseudocode for the presented algorithm can be found 
in \cite{robustlowrank} and implementation of the algorithm (as well as the midpoint-rule 
extension) in \cite{BUGimplementation}. 

\subsection{Discrete Conservation Equations}

We show that the integrator proposed satisfies the moment continuity equations 
\ref{eq:mass_continuity} - \ref{eq:energy_continuity} as well as preserves the 
electric energy up to first oder. 

\iffalse
\begin{lemma}\label{lem:conservation}
    Consider the $P$-th order moment $M_P^\iota$ with corresponding 
    multiindex $\varrho$ as in equation \ref{eq:multiindex}.
    Let $U : \Omega_v \to \bbR^m$ be chosen such that 

    \begin{equation}
        v \mapsto v^\varrho \in \spn \left\{ U_1, \ldots, U_m \right\} . 
    \end{equation}
 
    Let further $\widetilde{ X }$ be constructed for the time step $t$ as in equation 
    \ref{eq:conservative_projection}.  Then 

    \begin{equation}
        E(t) \cdot \int_{\Omega_v} \varrho v^{\varrho - 1} f(t) \,dv 
        \in \spn \left\{ \widetilde{ X }_k \right\}_k . 
    \end{equation}
\end{lemma}

\begin{proof}
    Let $\xi = \left( \xi_1, \ldots, \xi_m \right)$ be the coefficients of $v^\varrho$ 
    in the $U$ basis,

    \begin{equation}
        v^\varrho = \sum_\ell \xi_\ell\, U_\ell (v) . 
    \end{equation}

    By observing the Euler step equation for $K_P$ from equation 
    \ref{eq:K_galerkin} we notice

    \begin{equation}
        \begin{split}
            \underbrace{
                \sum_\ell \xi_\ell\, K_\ell (t + \tau)
            }_{
                \in \spn \left\{ \widetilde{ X }_k \right\}_k
            } 
            &= \sum_\ell \xi_\ell\, \left( 
                K_\ell (t) + \tau \left( V_\ell (t),\ RHS \right)_v 
            \right) \\
            &= \underbrace{
                    \sum_\ell \xi_\ell\, K_\ell (t) 
                    - \tau \sum_{j l} \xi_\ell\, c^1_{\ell j} \cdot \nabla_x K_j (t)
                }_{
                    \in \spn \left\{ \widetilde{ X }_k \right\}_k
                } 
                + \tau \sum_{\ell j} \xi_\ell\, c^2_{\ell j} \cdot E(t) K_j (t) 
        \end{split}
    \end{equation}

    for $c^1_{\ell j} = \left\langle V_\ell (t),\ vV_j (t) \right\rangle_v$, 
    $c^2_{\ell j} = \left(\, V_\ell (t),\ \nabla_v \left[ f_0 V_j (t) \right] \,\right)_v$. It follows 
    that 

    \begin{equation}
        \sum_{\ell j} \xi_\ell\, c^2_{\ell j} \cdot E(t) K_j (t) 
        = E(t) \cdot \left( \sum_{\ell j} \xi_\ell\, c^2_{\ell j} K_j (t) \right) 
        \in \spn \left\{ \widetilde{ X }_k \right\}_k . 
    \end{equation}

    But this can be rewritten as 

    \begin{equation}
        \begin{split}
            \sum_{\ell j} \xi_\ell\, c^2_{\ell j} K_j (t) 
            &= \sum_j \left( 
                \sum_\ell \xi_\ell\, U_\ell,\ 
                \nabla_v \left[ f_0 V_j (t) \right] 
            \right)_v \\
            &= \sum_j \left(\, v^\varrho,\ \nabla_v \left[ f_0 V_j (t) \right] \,\right)_v \\
            &= \sum_j \int_{\Omega_v} 
                v^\varrho\, 
                \nabla_v \left[ f_0 V_j (t) \right] \,dv\ 
                K_j (t) \\
            &= - \int_{\Omega_v} \varrho v^{\varrho - 1} \sum_j f_0 K_j (t) V_j (t) \,dv \\
            &= - \int_{\Omega_v} \varrho v^{\varrho - 1} f (t) \,dv . 
        \end{split}
    \end{equation}
\end{proof}
\fi

\begin{lemma}\label{lem:conservation}
    Let $\left( \widetilde{ X }_k \right)_k$ be constructed as in 
    \ref{eq:conservative_projection}. Then
    $\left( \widetilde{ X }_k \right)_k$ spans $RHS(\widetilde{ f })$, 
    that is, 

    \begin{equation}
        \sum_k \widetilde{ X }_k \left\langle 
            \widetilde{ X }_k,\ RHS(\widetilde{ f }) 
        \right\rangle_x
        = RHS(\widetilde{ f }) . 
    \end{equation}
\end{lemma}

\begin{proof}
    We note first that due to the augmentation of the basis, 
    $\widetilde{ f } = f = f_0 X^T S V$. 
    The claim now follows from a direct calculation using the fact that 
    $\widetilde{ X }$ spans a basis of $\nabla_x X$ and $E X$

    \begin{equation}
        \begin{split}
            \sum_k \widetilde{ X }_k \left\langle 
                \widetilde{ X }_k,\ 
                RHS(\widetilde{ f })
             \right\rangle_x
            &= \sum_{i j} S_{i j} \left( \sum_k \widetilde{ X }_k \left\langle 
                \widetilde{ X }_k,\ 
                \nabla_x X_i
             \right\rangle_x \right) \cdot v V_j \\ 
            &\quad\quad\quad\quad\quad\quad\quad\quad
             + \sum_{i j} S_{i j} \left( \sum_k \widetilde{ X }_k \left\langle 
                \widetilde{ X }_k,\ 
                E X_i
              \right\rangle_x \right) \cdot \nabla_v \left[ f_0 V_j \right] \\ 
            &= \sum_{i j} S_{i j} \nabla_x X_i \cdot v V_j 
                + \sum_{i j} S_{i j} E X_i \cdot \nabla_v \left[ f_0 V_j \right]
            = RHS(f) . 
        \end{split}
    \end{equation}
\end{proof}

\begin{lemma}
    Let $\widetilde{ K }$ be constructed as in \ref{eq:K_L_step}. Then 
    for $\ell \leq m$, 

    \begin{equation}
        \left( U_\ell, f (t + \tau) \right)_v
        = X(t + \tau)^T S(t + \tau) \left\langle U_\ell, V(t + \tau) \right\rangle_v
        = \widetilde{ K }_\ell . 
    \end{equation}
\end{lemma}

The above lemma effectively states that the truncation step is indeed 
conservative. This is crucial in the proceeding theorem. 

\begin{proof}
    Calculate

    \begin{equation}
        \begin{split}
            \left( U_\ell,\ f(t + \tau) \right)_v 
            &= \sum_{i j} X_i (t + \tau) S_{i j} (t + \tau) 
                \left\langle U_\ell,\ V_j \right\rangle_v \\ 
            &= \sum_i X_i (t + \tau) S_{i \ell} (t + \tau) \\
            &= \sum_i X_i (t + \tau) \sum_j R_{i j} \begin{bmatrix}
                S^{cons} & \\ 
                & S^{rem} 
            \end{bmatrix}_{j \ell}
        \end{split} . 
    \end{equation}

    Since $\ell \leq m$ (recall $V_\ell$ for $\ell > m$ was denoted $W_{\ell - m}$) 
    and $S^{cons} \in \bbR^{m \times m}$, 

    \begin{equation}
        \begin{split}
            \sum_i X_i (t + \tau) \sum_j R_{i j} \begin{bmatrix}
                S^{cons} & \\ 
                & S^{rem} 
            \end{bmatrix}_{j \ell}
            &= \sum_{j=1}^m \sum_i X_i (t + \tau) R_{i j} S^{cons}_{j \ell} \\ 
            &= \sum_{j=1}^m \widehat{ X }_j S^{cons}_{j \ell} \\ 
            &= \sum_i X^{cons}_i S^{cons}_{i \ell} \\ 
            &= \widetilde{ K }_\ell . 
        \end{split} 
    \end{equation}
\end{proof}

\begin{theorem}\label{thm:conservation}
    Consider the $P$-th order moment $M_P^\iota$ with corresponding 
    multiindex $\varrho$ as in equation \ref{eq:multiindex}. $M_P^\iota$ 
    satisfies the time-discrete continuity equation 

    \begin{multline}\label{eq:conservation_higher_order}
        \frac{M_{P+1}^\iota (t + \tau) - M_{P+1}^\iota (t)}{\tau} 
        + \nabla_x \cdot \int_{\Omega_v} v^{\varrho + 1} f(t) \,dv 
        + E(t) \cdot \int_{\Omega_v} \varrho v^{\varrho - 1} f(t) \,dv \\[2ex] 
        = \left( 
            \Pi_{(\spn U)^\perp} \left[ v^\varrho \right],\ 
            \frac{f(t + \tau) - f(t)}{\tau} - RHS (f(t))
        \right)_v 
    \end{multline}

    where 

    \begin{equation}
        \Pi_{(\spn U)^\perp} \left[ g \right] 
        = g - \sum_\ell U_\ell \left\langle U_\ell,\ g \right\rangle_v . 
    \end{equation}
\end{theorem}

\begin{remark}
    \begin{enumerate}
        \item In particular, when $U : \Omega_v \to \bbR^m$ is such that 

            \begin{equation}\label{eq:U_multiindex}
                v \mapsto v^\varrho \in \spn \left\{ U_1, \ldots, U_m \right\} 
            \end{equation}
            
            then the right hand side of equation \ref{eq:conservation_higher_order} 
            is zero. Hence, when equation \ref{eq:U_multiindex} holds for all 
            multiindices with $\left| \varrho \right| = P$, then the continuity 
            equation for the tensor $M_P$ holds. 
        \item The error representation in the form of an inner product is much 
            smaller than the norm-based error bounds typically given in numerical 
            analysis. Indeed, consider the scalar-valued error estimate

            \begin{equation}\label{eq:pessimistic_error}
                \begin{split}
                    &\left\| 
                        \frac{M_{P+1}^\iota (t + \tau) - M_{P+1}^\iota (t)}{\tau} 
                        + \nabla_x \cdot \int_{\Omega_v} v^{\varrho + 1} f(t) \,dv 
                        + E(t) \cdot \int_{\Omega_v} \varrho v^{\varrho - 1} f(t) \,dv
                    \right\|_x^2 \\ 
                    &= \left\| 
                        \left\langle 
                        \Pi_{(\spn U)^\perp} \left[ v^\varrho \right],\ 
                        \frac{f(t + \tau) - f(t)}{f_0 \tau} - \frac{1}{f_0} RHS (f(t))
                        \right\rangle_v 
                     \right\|_x^2 \\ 
                    &\leq \left\| \Pi_{(\spn U)^\perp} \left[ v^\varrho \right] \right\|_v \ 
                    \int_{\Omega_x} \left\| \frac{f(t + \tau) - f(t)}{f_0 \tau} - \frac{1}{f_0} RHS (f(t)) \right\|_v \,dx . 
                \end{split}
            \end{equation}

            where for the inequality we used the Cauchy-Schwarz inequality. 
            Indeed, the inequality is highly pessimistic, the latter term is 
            often multiple orders of magnitude larger than the former, see 
            section \ref{sec:landau}. 
    \end{enumerate}
\end{remark}

\begin{proof}
    We first note that the discrete continuity equation can be written as 

    \begin{equation}\label{eq:continuity_as_inner_product}
        \left( v^\varrho,\ \frac{f(t+\tau) - f(t)}{\tau} - RHS (f(t)) \right)_v . 
    \end{equation}

    Hence, the claim of the theorem is that when 
    $v \mapsto v^\varrho \in \spn \left\{ U_1, \ldots, U_m \right\}$, 
    the expression \ref{eq:continuity_as_inner_product} is zero. To that end, 
    split $v^\varrho$ into two components
    
    \begin{equation}\label{eq:split_v_varrho}
        v^\varrho = \sum_\ell \xi_\ell U_\ell (v) 
            + \left( \Pi_{(\spn U)^\perp} \left[ v^\varrho \right] \right) (v) . 
    \end{equation}
    
    We begin by giving a representation of \ref{eq:continuity_as_inner_product} 
    for the component of $v^\varrho$ in the span of $U$. 
    Recalling that 
    $\widetilde{ K }_\ell = \sum_k \widetilde{ X }_k \widetilde{ S }_{k \ell} (t + \tau)$ 
    and 
    $\widetilde{ S }_{k \ell} (t + \tau) 
    = \widetilde{ S }_{k \ell} + \tau \left( \widetilde{ X }_k U_\ell,\ RHS (\widetilde{ f }) \right)_{x v}$, 
    we may write 

    \begin{equation}\label{eq:K_tilde_ell}
        \begin{split}
            \widetilde{ K }_\ell &= 
            \sum_k \widetilde{ X }_k \left[ 
                \sum_{i j} \left\langle \widetilde{ X }_k,\ X_i (t) \right\rangle_x
                    S_{i j} (t) \left\langle V_\ell (t),\ \widetilde{ V }_j \right\rangle_v
                + \tau \left( \widetilde{ X }_k U_\ell,\ RHS (\widetilde{ f }) \right)_{x v}
             \right] \\
            &= \sum_i S_{i \ell} (t) \sum_k 
                \widetilde{ X }_k \left\langle \widetilde{ X }_k,\ X_i (t) \right\rangle_x
                + \tau \sum_k \widetilde{ X }_k 
                    \left( \widetilde{ X }_k U_\ell,\ RHS (\widetilde{ f }) \right)_{x v} . 
        \end{split}
    \end{equation}

    The first term in equation \ref{eq:K_tilde_ell} is equal to 

    \begin{equation}
        \sum_i S_{i \ell} (t) \sum_k 
            \widetilde{ X }_k \left\langle \widetilde{ X }_k,\ X_i (t) \right\rangle_x
        = \sum_i S_{i \ell} (t) X_i (t) 
        = K_\ell (t) 
        = \left( U_\ell,\ f(t) \right)_v . 
    \end{equation}

    Hence, 

    \begin{equation}
        \left( U_\ell,\ \frac{f(t + \tau) - f(t)}{\tau} \right)_v
        = \sum_k \widetilde{ X }_k 
        \left( \widetilde{ X }_k U_\ell,\ RHS (\widetilde{ f }) \right)_{x v} . 
    \end{equation}

    Using the splitting in equation \ref{eq:split_v_varrho}, 

    \begin{equation}\label{eq:continuity_finite_diff_part}
        \left( v^\varrho,\ \frac{f(t + \tau) - f(t)}{\tau} \right)_v
        = \left( \sum_\ell \xi_\ell U_\ell,\ \frac{f(t + \tau) - f(t)}{\tau} \right)_v
            + \left( \Pi_{(\spn U)^\perp} \left[ v^\varrho \right],\ 
            \frac{f(t + \tau) - f(t)}{\tau} \right)_v . 
    \end{equation}

    The first term is rewritten to 

    \begin{equation}\label{eq:continuity_U_part}
        \begin{split}
            \left( \sum_\ell \xi_\ell U_\ell,\ \frac{f(t + \tau) - f(t)}{\tau} \right)_v
            &= \sum_k \widetilde{ X }_k \left( 
                \widetilde{ X }_k \sum_\ell \xi_\ell U_\ell,\ 
                RHS(\widetilde{ f }) 
            \right)_{x v} \\ 
            &= \sum_k \widetilde{ X }_k 
                \left( \widetilde{ X }_k v^\varrho,\ RHS (\widetilde{ f }) \right)_{x v} \\
                &\quad\quad\quad\quad
                - \sum_k \widetilde{ X }_k \left( \widetilde{ X }_k \Pi_{(\spn U)^\perp} \left[ v^\varrho \right],\ 
                RHS( \widetilde{ f } ) \right)_{x v} \\ 
            &= \left( 
                v^\varrho - \Pi_{(\spn U)^\perp} \left[ v^\varrho \right],\ 
                \sum_k \widetilde{ X }_k \left\langle \widetilde{ X }_k,\ RHS(\widetilde{ f }) \right\rangle_x
             \right)_v \\ 
            &= \left( v^\varrho - \Pi_{(\spn U)^\perp} \left[ v^\varrho \right],\ RHS(f) \right)_v 
        \end{split}
    \end{equation}

    where for the last equality we have used lemma \ref{lem:conservation}. Inserting 
    equation \ref{eq:continuity_U_part} into equation \ref{eq:continuity_finite_diff_part} 
    and rearranging yields the claim. 

    \iffalse
    Due to the augmentation of the basis $\widetilde{ X }$, the projection on the right side 
    of this inner product acts as an identity. Therefore 

    \begin{equation}
        \left( \sum_\ell \xi_\ell U_\ell,\ \frac{f(t + \tau) - f(t)}{\tau} \right)_v
        = 
    \end{equation}

    \begin{equation}
        \begin{split}
            \left( v^\varrho,\ \frac{f(t + \tau) - f(t)}{\tau} \right)_v
            &= \left( \sum_\ell \xi_\ell U_\ell,\ \frac{f(t + \tau) - f(t)}{\tau} \right)_v
                + \left( \Pi_{(\spn U)^\perp} \left[ v^\varrho \right],\ 
                \frac{f(t + \tau) - f(t)}{\tau} \right)_v \\ 
            &= \sum_k \widetilde{ X }_k 
            \left( \widetilde{ X }_k v^\varrho,\ RHS (\widetilde{ f }) \right)_{x v}
            + \left( \Pi_{(\spn U)^\perp} \left[ v^\varrho \right],\ 
                \frac{f(t + \tau) - f(t)}{\tau} \right)_v
            - \sum_k \widetilde{ X }_k \left( \widetilde{ X }_k \Pi_{(\spn U)^\perp} \left[ v^\varrho \right],\ 
            RHS( \widetilde{ f } ) \right)
            %= \sum_k \widetilde{ X }_k 
            %\left( \widetilde{ X }_k v^\varrho,\ RHS (\widetilde{ f }) \right)_{x v} 
        \end{split}
    \end{equation}

    Thanks to the augmentation of the bases, this can be written as 

    \begin{equation}\label{eq:varPsi}
        \begin{split}
            \sum_k \widetilde{ X }_k &
                \left( \widetilde{ X }_k v^\varrho,\ RHS (\widetilde{ f }) \right)_{x v} \\
            &= \sum_k \widetilde{ X }_k 
            \left( \widetilde{ X }_k v^\varrho,\ 
                RHS (
                    f = f_0 
                        \widetilde{ X }^T 
                        \widetilde{ M }^T 
                        S (t) 
                        \widetilde{ N }
                        \widetilde{ V }
                ) 
            \right)_{x v} \\
            &= \sum_k \widetilde{ X }_k 
            \left( \widetilde{ X }_k v^\varrho,\ 
                RHS ( f = f_0 X(t)^T S (t) V(t) ) \right)_{x v} . 
        \end{split}
    \end{equation}

    We have 

    \begin{equation}\label{eq:X_k_v_rho}
        \begin{split}
            &\left( 
                \widetilde{ X }_k \,v^\varrho,\ RHS ( f = f_0 X(t)^T S(t) V(t) )
            \right)_{x v} 
            = \left( 
                \widetilde{ X }_k \,v^\varrho,\ RHS ( f(t) )
            \right)_{x v} \\
            &= \sum_{i j} S_{i j} 
                \left\langle \widetilde{ X }_k,\ \nabla_x X_i (t) \right\rangle_x \cdot 
                \left( v^{\varrho + 1},\ f_0 V_j (t) \right)_v \\ 
                &\quad\quad\quad\quad\quad\quad\quad%\quad\quad 
                + \sum_{i j} S_{i j} 
                \left\langle \widetilde{ X }_k,\ E(t) X_i (t) \right\rangle_x \cdot 
                \left( v^\varrho,\ \nabla_v \left[ f_0 V_j (t) \right] \right)_v \\ 
            &= \sum_{i j} S_{i j} 
                \left\langle \widetilde{ X }_k,\ \nabla_x X_i (t) \right\rangle_x \cdot 
                \left\langle v^{\varrho + 1},\ V_j (t) \right\rangle_v \\ 
                &\quad\quad\quad\quad\quad\quad\quad%\quad\quad 
                + \sum_{i j} S_{i j} 
                \left\langle \widetilde{ X }_k,\ E(t) X_i (t) \right\rangle_x \cdot 
                \left\langle \varrho v^{\varrho - 1},\ V_j (t) \right\rangle_v . 
        \end{split}
    \end{equation}

    where for the last equality we used integration by parts. 
    Inserting \ref{eq:X_k_v_rho} into \ref{eq:varPsi} and noting that $\widetilde{ X }$ 
    forms a basis of $\left\{ X_i (t) \right\}_i$ and $\left\{ \nabla X_i (t) \right\}_i$ 
    and using lemma \ref{lem:conservation} yields 

    \begin{equation}
        \begin{split}
            \sum_k \widetilde{ X }_k &
            \left( \widetilde{ X }_k v^\varrho,\ 
                RHS ( f(t) ) \right)_{x v} \\
            &= \int_{\Omega_v} \sum_{i j} 
                f_0\, V_j (t) S_{i j} (t) \nabla_x X_i (t) \cdot v^{\varrho + 1} \,dv \\ 
                &\quad\quad\quad\quad\quad\quad\quad%\quad\quad 
                + \int_{\Omega_v} \sum_{i j} 
                f_0\, V_j (t) S_{i j} (t) X_i (t) E(t) \cdot \varrho v^{\varrho - 1} \,dv \\ 
            &= \nabla_x \cdot \int_{\Omega_v} v^{\varrho + 1} 
                \sum_{i j} f_0\, V_j (t) S_{i j} (t) X_i (t) \,dv \\ 
                &\quad\quad\quad\quad\quad\quad\quad%\quad\quad 
                + E(t) \cdot \int_{\Omega_v} \varrho v^{\varrho - 1} 
                \sum_{i j} f_0\, V_j (t) S_{i j} (t) X_i (t) \,dv \\ 
            &= \nabla_x \cdot \int_{\Omega_v} v^{\varrho + 1} f(t) \,dv
                + E(t) \cdot \int_{\Omega_v} \varrho v^{\varrho - 1} f(t) \,dv \\ 
            &= \left( v^\varrho,\ RHS(f(t)) \right)_v
        \end{split}
    \end{equation}

    which was to be shown. 
    \fi
\end{proof}

Due to the orthonormality condition on $U$, the most reasonable choice to satisfy theorem 
\ref{thm:conservation} is given by Hermite polynomials, that is, we choose tensor 
products of Hermite polynomials up to degree $P$. Setting $P = 0$ implies conservation of 
mass and $P = 1$ implies conservation of momentum. To see conservation of total energy, 
a final calculation must be made. 

\begin{corollary}\label{thm:electric_energy}
    Let $U : \Omega_v \to \bbR^m$ be chosen such that 

    \begin{equation}
        v \mapsto v^2 \in \spn \left\{ U_1, \ldots, U_m \right\} 
    \end{equation}

    Assume that the numerically computed electric field satisfies 

    \begin{equation}
        \left\| E(t + \tau) - E(t) \right\| = \mathcal{O} (\tau^\mu)
    \end{equation}
    
    for some $\mu \geq 1$. Then the time-discrete 
    continuity equation for total energy is satisfied to order $2\mu - 1$, that is,

    \begin{equation}
        \frac{e(t + \tau) - e(t)}{\tau}
        + \frac{1}{2} \nabla_x \cdot (\trace M_3 (t)) 
        + (M_1 (t) - \frac{E(t + \tau) - E(t)}{\tau}) \cdot E (t)
        = \mathcal{O} (\tau^{2\mu - 1}) . 
    \end{equation}

\end{corollary}

\begin{proof}
    Note that 

    \begin{equation}\label{eq:discrete_1}
        \frac{e(t + \tau) - e(t)}{\tau} 
        = \frac{1}{2} \frac{\trace M_2 (t + \tau) - \trace M_2 (t)}{\tau} 
            + \frac{1}{2} \frac{E(t + \tau)^2 - E(t)^2}{\tau} ,
    \end{equation}

    the first term of which is equal to 
    
    \begin{equation}\label{eq:discrete_2}
        \frac{\trace M_2 (t + \tau) - \trace M_2 (t)}{\tau} 
        = - \nabla_x \cdot ( \trace M_3 (t) ) - 2 M_1 (t) \cdot E (t)
    \end{equation}

    by theorem 
    \ref{thm:conservation}. The second term can be rewritten as 

    \begin{equation}\label{eq:discrete_3}
        \begin{split}
            \frac{E(t + \tau)^2 - E(t)^2}{\tau} 
            &= \frac{(E(t + \tau) - E(t))^2 + 2 E(t + \tau) \cdot E(t) - 2 E(t)^2}{\tau} \\ 
            &= 2 E(t) \cdot \frac{E(t +\tau) - E(t)}{\tau} 
                + \frac{(E(t + \tau) - E(t))^2}{\tau} . 
        \end{split}
    \end{equation}

    Since $E(t + \tau)$ is always an $\mathcal{O} (\tau^\mu)$ away from $E(t)$, the second 
    term is an $\mathcal{O} (\tau^{2\mu - 1})$. Inserting the expressions 
    \ref{eq:discrete_2} and \ref{eq:discrete_3} into equation \ref{eq:discrete_1} now 
    yields the claim. 
\end{proof}

Theorem \ref{thm:conservation} states that when all monomials up to order $P$ are held 
constant in $U$, then the corresponding continuity equations up to order $P$ are 
satisfied. For higher order continuity equations, we can similarly guarantee a worst-case 
bound on the error, dependent on how well the discrete differential 
$\frac{1}{\tau} (f(t + \tau) - f(t))$ matches $RHS(f)$ in the direction of some monomial 
$v \mapsto v^\varrho$. 

% -------------------------------------------------------------------------------------- %
